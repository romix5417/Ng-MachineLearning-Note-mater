\chapter{单变量线性回归}

\begin{flushleft}
故不登高山,不知天之高也;不临深溪,不知地之厚也;不闻先王之遗言,不知学问之大也。
\end{flushleft}

\begin{flushright}
---荀子   
\end{flushright}

\section{模型描述}
 我们目前都是学生,接触 \LaTeX{} 的时间也不是很长,因此,对于此模板的错误还请多多包涵! 目前,模板的拓展性或者可移植性有待完善,所以,我们强烈建议用户不要大幅修改模板文件,我们的初衷是提供一套模板,让初学者能够使用一些比较美观,优雅的模板。如果在使用过程中,想修改一些简单的东西需要帮忙,请联系我们,我们的邮箱是:\href{elegantlatex2e@gmail.com}{elegantlatex2e@gmail.com}。我们将竭尽全力提供帮助!

值此版本发行之际,我们 Elegant\LaTeX{} 项目组向大家重新介绍一下我们的工作,我们的主页是 \href{http://elegantlatex.tk}{http://elegantlatex.tk},我们这个项目致力于打造一系列美观、优雅、简便的模板方便使用者记录学习历史。其中目前在实施或者在规划中的子项目有 书籍模板ElegantBook、笔记模板ElegantNote、幻灯片模板 ElegantSlide。这些子项目的名词是一体的,请在使用这些名词的时候不要将其断开(如 Elegant Note是不正确的写法)。并且,Elegant\LaTeX{}  Book 指的即是 ElegantBook。

\section{代价函数}

基于本模板追求视觉上的美观的角度,强烈建议使用者安装./fonts/文件夹下的字体。出于版权的考虑,务必不能将此模板用于涉及盈利目的的商业行为,否则,后果自负,本模板带的字体仅供学习使用,如果您喜欢某种字体,请自行购买正版。本文主要使用的字体如下
\begin{itemize}
\itemsep=3pt
\parskip=0pt
\item Adobe Garamond Pro
\item Minion Pro \& Myriad Pro  \& Inconsolata
\item 方正字体
\item 华文中宋
\end{itemize}

\begin{note}
中文正文使用了华文中宋,Minion Pro为英文衬线字体(\verb|\rmfamily|),Myriad Pro为英文非衬线字体(\verb|\sffamily|),Inconsolata 为英文打字机字体(\verb|\ttfamily|)。

并且,如果系统内安装了Adobe 字体,大家可以把模板中使用到的黑体,楷体,宋体等字替换成 Adobe 字体,这样可以达到最佳效果。
\end{note}

\section{代价函数(一)}
\subsection{编译方式}
本模板基于book文类,所以book的选项对于本模板也是有效的。但是,只支持 \XeLaTeX{},编码为 UTF-8,推荐使用 \TeX{}live编译。作者编写环境为Win8.1(64bit)+\TeX{}live 2013,由于使用了参考文献,所以,编译顺序为\XeLaTeX->\BibTeX->\XeLaTeX->\XeLaTeX。

本文特殊选项设置共有3类,分为{\color{main}颜色} 、{\color{main} 数学字体 }以及{\color{main} 章标题显示风格}。

\subsection{选项设置}
第一类为{\color{main}颜色}主题设置,内置 3 组颜色主题,分别为 \verb|green|(默认),\verb|cyan|,\verb|blue|,另外还有一个自定义的选项 \verb|nocolor|,用户\textbf{必须}在使用模板的时候选择某个颜色主题或选择 \verb|nocolor|选项。调用颜色主题 \verb|green| 的方法为\verb|\documentclass[green]{elegantbook}|或者使用\verb|\documentclass[color=green]{elegantbook}|。需要改变颜色的话请选择 \verb|nocolor|选项或者使用 \verb|color=none|,然后在导言区定义main、seco、thid颜色,具体的方法如下:
\begin{verbatim}
\definecolor{main}{RGB}{70,70,70}    %定义main颜色值
\definecolor{seco}{RGB}{115,45,2}    %定义seco颜色值
\definecolor{thid}{RGB}{0,80,80}     %定义thid颜色值
\end{verbatim}

第二类为{\color{main}数学字体}设置,有两个可选项,分别是 mathpazo(默认) 和 mtpro2 字体,调用mathpazo 字体使用 \verb|\documentclass[mathpazo]{elegantbook}|,调用 mtpro2 字体时需要把 \verb|mathpazo|换成 \verb|mtpro|,mathpazo 不需要用户自己安装字体,mtpro2 的字体需要自己安装。

\begin{table}[htp]
\centering
\begin{tabular}{ccccc}
\toprule
	  & green & cyan & blue & 主要使用的环境\\
\midrule
main & \makecell{{\color{main1}\rule{1cm}{1cm}}}& \makecell{{\color{main2}\rule{1cm}{1cm}}}&\makecell{ {\color{main3}\rule{1cm}{1cm}}}& newthem \ newlemma \ newcorol\\

seco &\makecell{ {\color{seco1}\rule{1cm}{1cm}}}& \makecell{{\color{seco2}\rule{1cm}{1cm}}}&\makecell{ {\color{seco3}\rule{1cm}{1cm}}}&newdef\\

thid &\makecell{ {\color{thid1}\rule{1cm}{1cm}}}& \makecell{{\color{thid2}\rule{1cm}{1cm}}}&\makecell{ {\color{thid3}\rule{1cm}{1cm}}}&newprop\\
\bottomrule
\end{tabular}
\caption{ElegantBook 模板中的三套颜色主题\label{tab:color thm}}
\end{table}

第三类为{\color{main} 章标题显示风格},包含 \verb|hang|(默认)与 \verb|display|两种风格,区别在于章标题单行显示(\verb|hang|)与双行显示(\verb|display|),本说明使用了 \verb|hang|。调用方式为 \verb|\documentclass[hang]{elegantbook}|或者 \verb|\documentclass[titlestyle=hang]{elegantbook}|。

综合起来,同时调用三个选项使用 \verb|\documentclass[color=X,Y,titlestyle=Z]{elegantbook}|。其中 \verb|X| 可以选择 \verb|green|,\verb|cyan|,\verb|blue|,\verb|none|;\verb|Y| 可以选择 \verb|mathpazo|或者 \verb|mtpro|;\verb|Z|可以选择 \verb|hang| 或者 \verb|display|。

\subsection{数学环境简介}
在我们这个模板中,定义了三大类环境
\begin{enumerate}
\item 定理类环境,包含标题和内容两部分。根据格式的不同分为3种
\begin{itemize}
\item {\color{main} newthem、newlemma、newcorol} 环境,颜色为主颜色main,三者编号均以章节为单位;
\item {\color{main} newdef} 环境,含有一个可选项,编号以章节为单位,颜色为seco;
\item {\color{main} newprop} 环境,含有一个可选项,编号以章节为单位,颜色为thid。
\end{itemize}
\item 证明类环境,有  {\color{main} newproof、note、remark、solution} 环境,特点是,有引导符和引导词,并且 newproof、solution 环境有结束标志。
\item 结论类环境,有 {\color{main}  conclusion、assumption、property} 环境,三者均以粗体的引导词为开头,和普通段落格式一致。
\item 示例类环境--- {\color{main} example、exercise}环境,编号以章节为单位,其中exercise环境有引导符。
\item 自定义环境--- {\color{main} custom},带一个必选参数,格式与conclusion 环境很类似。
\end{enumerate}

\subsection{可编辑的字段}
在模板中,可以编辑的字段分别为作者\verb|\author|、\verb|\email|、\verb|\zhtitle|、\verb|\zhend|、\verb|\entitle|、\verb|\enend|、\verb|\version|。并且,可以根据自己的喜好把封面水印效果的\verb|cover.pdf|替换掉,以及封面中用到的\verb|logo.pdf|。

\section{代价函数(二)}

\section{梯度下降}

\section{梯度下降知识点总结}

\section{线性回归的梯度下降}

\section{本章课程总结}