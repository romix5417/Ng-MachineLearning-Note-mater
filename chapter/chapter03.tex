\chapter{线性回归回顾}

\begin{center}
为往圣继绝学,为万世开太平!
\end{center}

\begin{flushright}
---张载  
\end{flushright}

\section{矩阵和向量}
\lipsum[3]
考虑如下的随机动态规划问题
\begin{align*}
&\max(\min)\quad \mathbb{E}\int_{t_0}^{t_1}f(t,x,u)\,dt\\
&\quad\mbox{s.t.} \quad dx=g(t,x,u)dt+\sigma(t,x,u)dz\\
&\quad \hspace{2.em} k(0)=k_0\;\text{given}
\end{align*}

where $z$ is stochastic process or white noise or wiener process.

\begin{newdef}[Wiener Process]
If $z$ is wiener process, then for any partition $t_0,t_1,t_2,\ldots$ of time interval, the random variables $z(t_1)-z(t_0),z(t_2)-z(t_1),\ldots$ are independently and normally distributed with zero means and variance $t_1-t_0,t_2-t_1,\ldots$
\end{newdef}

\lipsum[5]

\begin{example}
$E$ and $F$ be two events such that $\mbf{P}(E)=\mbf{P}(F)=1/2$, and $\mbf{P}(E\cap F)=1/3$, let $\mathscr{F}=\sigma(Y)$,  $X$ and $Y$ be the indicate function of $E$ and $F$ respectively. How to compute $\mathbb{E}[ X\mid \mathscr{F} ]$?
\end{example}
\lipsum[4]
\begin{exercise}
let $S=l^\infty=\big\{(x_n)\mid \exists\, M \text{ such that } \forall n, |x_n|\leq M,x_n\in \mathbb{R}\big\}$, $\rho_{\infty}(x,y)=\sup\limits_{n\geq 1}|x_n-y_n|$, show that $\big(l^\infty,\rho_{\infty}\big)$ is complete.
\end{exercise}

\begin{newthem}[勾股定理]
勾股定理的数学表达(Expression)为
\[a^2+b^2=c^2\]
其中$a,b$为直角三角形的两条直角边长,$c$为直角三角形斜边长。
\end{newthem}

\begin{note}
在本模板中,引理(lemma),推论(corollary )的样式和定理的样式一致,包括颜色,仅仅只有计数器的设置不一样。在这个例稿中,我们将不给出引理推论的例子。
\end{note}


\lipsum[4]

\begin{newprop}[最优性原理]
如果$u^*$在$[s,T]$上为最优解,则$u^*$在$[s,T]$任意子区间都是最优解,假设区间为$[t_0,t_1]$的最优解为$u^*$,则$u(t_0)=u^{*}(t_0)$,即初始条件必须还是在$u^*$上。
\end{newprop}

\lipsum[5-6]
\begin{newcorol}
假设$V(\cdot,\cdot)$为值函数,则跟据最大值原理,有如下推论
\[
V(k,z)=\max\Big\{u\big(zf(k)-y\big)+\beta \mathbb{E}V(y,z^\prime)\Big\}
\]
\end{newcorol}

\begin{newproof}
因为 $y^*=\alpha\beta z k^\alpha$,$V(k,z)=\alpha/1-\alpha\beta\ln k_0+1/1-\alpha\beta \ln z_0+\Delta$。
\begin{align*}
\text{右边}&=\Big\{u\big(zf(k)-y\big)+\beta \mathbb{E}V(y,z^\prime)\Big\}\\
&=\ln(zk^\alpha-\alpha\beta zk^\alpha)+\beta\mathbb{E}\Big[\frac{\alpha}{1-\alpha\beta}\ln y+\frac{1}{1-\alpha\beta}\ln z^\prime+\Delta\Big]\\
&=\ln(1-\alpha\beta)zk^\alpha+\beta\Big\{\mathbb{E}\big[\frac{\alpha}{1-\alpha\beta}\ln \alpha\beta z k^\alpha\big]+\frac{1}{1-\alpha\beta}\mathbb{E}[\ln z^\prime]+\Delta\Big\}
\end{align*}
利用$\mathbb{E}[\ln z^\prime]=0$,并将对数展开得
\begin{align*}
\text{右边}&=\ln (1-\alpha\beta)+\ln z+\alpha\ln k+\frac{\alpha\beta}{1-\alpha\beta}\big[\ln \alpha\beta+\ln z+\alpha\ln k\big]+\frac{\beta}{1-\alpha\beta}\mu+\beta \Delta\\
&=\frac{\alpha}{1-\alpha\beta}\ln k+\frac{1}{1-\alpha\beta}\ln z+\Delta
\end{align*}
所以$\text{左边}=\text{右边}$,证毕。
\end{newproof}



\begin{property}
Properties of Cauchy Sequence
\begin{enumerate}\parskip=0pt \itemsep=0pt
\item $\{x_k\}$ is cauchy sequence then $\{x_k^i\}$ is cauchy sequence.
\item $x_k\in \mathbb{R}^n$, $\rho(x,y)$ is Euclidean, then cauchy is equivalent to convergent, $(\mathbb{R}^n,\rho)$ metric space is complete.
\end{enumerate}
\end{property}


\lipsum[7]

\begin{custom}{Application}
This is one example of the custom environment, the key word is given by the option of custom environment.
\end{custom}


\lipsum[6]
\begin{newdef}[Contraction mapping]
$(S,\rho)$ is the metric space, $T: S\to S$, If there exists $\alpha\in(0,1)$ such that for any $x$ and $y\in S$, the distance
\begin{equation}
\rho(Tx,Ty)\leq \alpha\rho(x,y)
\end{equation}
Then $T$ is a {\color{main} contraction mapping}.
\end{newdef}

\begin{remark}
\begin{enumerate}
\parskip=0pt \itemsep=0pt
\item $T:S\to S$, where $S$ is a metric space, if  for any $x,y\in S$, $\rho(Tx,Ty)<\rho(x,y)$ is not contraction mapping.
\item Contraction mapping is continuous map.
\end{enumerate}
\end{remark}


\begin{conclusion}
看到一则小幽默,是这样说的:{\color{main} 别人都关心你飞的有多高,只有我关心你的翅膀好不好吃!}说多了都是泪啊!
\end{conclusion}

\section{加法和标量乘法}

\section{矩阵向量乘法}

\section{矩阵乘法}

\section{矩阵乘法特征}

\section{逆和转制}